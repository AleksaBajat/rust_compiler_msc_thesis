\section{Међурепрезентације изворног кода}

Приликом компајлирања \verb|Rust| изворног кода, \verb|Rust| \verb|frontend| мора да осигура да је 
код синтактички и семантички тачан да би могао да изгенерише правилну \verb|LLVM| међу репрезентацију 
на основу које се генерише машински код. Да би се ово ефективно извело \verb|Rust| \verb|frontend| поседује 
сопствене међурепрезентације које обављају дистинктне дужности. 

Лексички анализатор или лексер је улазна тачка компајлера. Лексер чита изворни код као ток карактера и групише карактере 
у секвенце које имају значење. Овај процес се назива лексичка анализа или скенирање, а секвенце карактера се називају лексеме. 
За сваку лексему лексички анализатор производи токен. Токени се обично састоје од типа и вредности.
Лексички анализатор је обично најједноставнији део компајлера за имплементирати. 
Уобичајно је да се лексери програмских језика имплементирају као коначни аутомат уз помоћ алата као што 
је \verb|Flex|. Ипак, у \verb|Rust| језику лексер је имплементиран ручно, што омогућава фину грануларност 
и високу контролу над процесом токенизације. Имајући у виду да је један од циљева \verb|Rust| језика 
да поседује веома детаљне и корисне поруке о грешци, ово је врло промишљена имплементациона одлука. 
Ток токена није нарочито корисна форма за анализу.

Наредни корак у процесу компаљирања јесте парсер који врши синтаксну анализу или парсирање. 
Парсер користи токене које је изгенерисао лексер да би креирао структуру облика стабла која се назива 
апстрактно синтаксно стабло. Апстрактно синтаксно стабло експлицитно назначава првенство оператора,
и релативно га је лако валидирати.

Последњи корак у \verb|Rust| \verb|frontend|-у јесте семантичка анализа. Семантичка анализа се спроводи 
кроз три међурепрезентације, репрезентација високог нивоа, типизирана репрезентација високог нивоа
и репрезентација средњег нивоа. Битан део семантичке анализе јесте провера типова где компајлер проверава 
да ли сваки оператор има операнде који се поклапају. У другим језицима често је постојање имплицитних 
конверзија једног типа у други, ова функционалност није постојана у \verb|Rust| језику. Свака конверзија 
мора бити експлицитно наведена од стране корисника.

У типично структуираном компајлеру свака од претходно наведених међурепрезентација кода представља 
посебан програм и засебно извршавање. Овакви компајлери се називају 
компајлери засновани на проласцима. \verb|Rust|-ов компајлер је у транзицији између 
компајлера занованог на проласцима и компајлера заснованог на потражњи. Компајлер 
заснован на потражњи уместо да ради на принципу архитектуре са цевима (\verb|pipeline|)
користи упите (\verb|query|) над изворним кодом и симултоно извршава целокупан процес.

Извршавање упита је мемоизовано. Први пут када се упит позове, извршиће се компутације 
везане за тај упит. Наредни позив истог упита резултат добавља из речника.
Овакав принцип извршавања се савршено уклапа у оквир инкременталне компилације. 
Резултат претходне компилације се чува у масовној меморији (складишту) који се може искористи
у тренутној компилацији.

Тренутно се свака репрезентација изворног кода на \verb|frontend|-у, 
након креирања апстрактног синтаксног стабла, заснива на упитима. 
Лексер и парсер још увек функционишу на основу пролазака, 
али циљ је да се и они рефакторишу како би користили упите.

Улазна тачка компајлера у \verb|rustc| изворном коду се налази у \verb|rustc_driver_impl| \verb|crate|-у
који поседује функцију \verb|run_compiler|. Ова функција даје веома добар преглед тока компајлирања 
и значајна је тачка прликом упознавања са изворним кодом. 