\section{Zaključak}

Ovaj master rad pružio je uvid u arhitekturu Rust kompajlera, sa posebnim fokusom na ključnu ulogu koju igraju međureprezentacije izvornog koda u procesu prevođenja. 
Počevši od osnova Rust jezika, njegovog ekosistema sa alatom Cargo, i oslanjanja na LLVM projekat za generisanje optimizovanog mašinskog koda, 
analiziran je sam proces kompajliranja.

Centralni deo rada detaljno je istražio putanju koda kroz seriju međureprezentacija. 
Pokazano je kako se od inicijalnog toka tokena i ASS-a, preko MVN-a koji vrši pojednostavljenje sintakse i predstavlja platformu za prelazak sa \verb|pipeline|
kompajlerske arhitekture na arhitekturu zasnovanu na upitima, dolazi do TMVN-a gde se integrišu informacije o tipovima ključne za bezbednost 
i analizu šablona.

Posebna pažnja posvećena je međureprezentaciji srednjeg nivoa (MSN). Naglašena je njena uloga kao eksplicitnog grafa kontrole toka,
što je čini fundamentalnom za implementaciju Rust-ovih garancija memorijske bezbednosti kroz proveru pozajmljivanja i neleksičke životne vekove.
Takođe, MSN služi kao osnova za važne Rust-specifične optimizacije, pre nego što se dalja optimizacija i generisanje koda prepuste LLVM-u.

Kroz ovu analizu, demonstrirano je da višefazni pristup sa pažljivo dizajniranim međureprezentacijama nije samo tehnički detalj,
već suštinski mehanizam koji omogućava Rustu da istovremeno postigne svoje glavne ciljeve: memorijsku bezbednost bez sakupljača smeća,
visoke performanse koje pariraju C/C++-u, i izražajnost jezika sa modernim karakteristikama. Složenost kompajlera je efikasno modularizovana,
omogućavajući rigorozne provere i optimizacije na najpogodnijem nivou apstrakcije. Razumevanje ovih internih faza ključno je za dalje unapređenje kompajlera,
razvoj alata za analizu koda i dublje shvatanje samog jezika Rust.