\section{Закључак}

Овај мастер рад пружио је увид у архитектуру \verb|Rust| компајлера, са посебним фокусом на кључну улогу коју играју међурепрезентације изворног кода у процесу превођења. 
Почевши од основа \verb|Rust| језика, његовог екосистема са алатом \verb|Cargo|, и ослањања на ЛЛВМ пројекат за генерисање оптимизованог машинског кода, 
анализиран је сам процес компајлирања.

Централни део рада детаљно је истражио путању кода кроз серију међурепрезентација. 
Показано је како се од иницијалног тока токена и АСС-а, преко МВН-а који врши поједностављење синтаксе и представља платформу за прелазак са \verb|pipeline|
компајлерске архитектуре на архитектуру засновану на упитима, долази до ТМВН-а где се интегришу информације о типовима кључне за безбедност 
и анализу шаблона.

Посебна пажња посвећена је међурепрезентацији средњег нивоа (МСН). Наглашена је њена улога као експлицитног графа контроле тока,
што је чини фундаменталном за имплементацију \verb|Rust|-ових гаранција меморијске безбедности кроз проверу позајмљивања и нелексичке животне векове.
Такође, МСН служи као основа за важне \verb|Rust|-специфичне оптимизације, пре него што се даља оптимизација и генерисање кода препусте ЛЛВМ-у.

Кроз ову анализу, демонстрирано је да вишефазни приступ са пажљиво дизајнираним међурепрезентацијама није само технички детаљ,
већ суштински механизам који омогућава \verb|Rust|у да истовремено постигне своје главне циљеве: меморијску безбедност без сакупљача смећа,
високе перформансе које парирају \verb|C|/|verb|C++|-у, и изражајност језика са модерним карактеристикама. Сложеност компајлера је ефикасно модуларизована,
омогућавајући ригорозне провере и оптимизације на најпогоднијем нивоу апстракције. Разумевање ових интерних фаза кључно је за даље унапређење компајлера,
развој алата за анализу кода и дубље схватање самог језика \verb|Rust|.  