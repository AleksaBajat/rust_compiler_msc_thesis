\section{Додатак 1}

\begin{listing}[H]
\begin{minted}{rust}
#[proc_macro_attribute]
pub fn expand_body(_attr: TokenStream, _item: TokenStream) -> TokenStream {
    let fun = format!("fn missing_body() 
    {{ println!(\"I'm not missing body anymore.\"); }}");

    let stream: TokenStream = fun.parse().unwrap();

    stream
}

#[expand_body]
fn missing_body();

fn main() {
    missing_body();
}
\end{minted}
\caption{Додавање тела функције уз помоћ макроа}
\label{lst:bonus_body_expand}
\end{listing}

\clearpage

\section{Додатак 2}
\begin{listing}[H]
\begin{minted}{rust}
fn try_mark_green(tcx, current_node) -> bool {
// Fetch the inputs to `current_node`, i.e. get the nodes that the direct
// edges from `node` lead to.
let dependencies = tcx.dep_graph.get_dependencies_of(current_node);
// Now check all the inputs for changes
for dependency in dependencies {
match tcx.dep_graph.get_node_color(dependency) {
Green => {
// This input has already been checked before and it has not
// changed; so we can go on to check the next one
}
Red => {
// We found an input that has changed. We cannot mark
// `current_node` as green without re-running the corresponding query.
    return false
}
Unknown => {
// This is the first time we look at this node. Let's try
// to mark it green by calling try_mark_green() recursively.
    if try_mark_green(tcx, dependency) {
// We successfully marked the input as green, on to the next.
    } else {
// We could *not* mark the input as green. This means we
// don't know if its value has changed. In order to find
// out, we re-run the corresponding query now!
        tcx.run_query_for(dependency);
// Fetch and check the node color again. Running the query
// has forced it to either red (if it yielded a different
// result than we have in the cache) or green (if it
// yielded the same result).
        match tcx.dep_graph.get_node_color(dependency) {
            Red => {
// The input turned out to be red, so we cannot mark `current_node` as green.
                return false
            }
            Green => {
// Re-running the query paid off! The result is the same as before, 
// so this particular input does not invalidate `current_node`.
            }
            Unknown => {
// There is no way a node has no color after re-running the query.
                panic!("unreachable") } } } } } }
tcx.dep_graph.mark_green(current_node);
true }
\end{minted}
\caption{Имплементација алгоритма "try-mark-green"}
\label{lst:bonus_try_mark_green}
\end{listing}

\section{Додатак 3}

\begin{listing}[H]
\begin{minted}{rust}
let Ok((thir, expr)) = tcx.thir_body(def) else { return };
// Runs all other queries that depend on THIR.
tcx.ensure_with_value().mir_built(def);
let thir = &thir.steal();
// If `thir` is empty, a type error occurred, skip this body.
if thir.exprs.is_empty() {
    return;
}

let hir_id = tcx.local_def_id_to_hir_id(def);
let safety_context = tcx.hir().fn_sig_by_hir_id(hir_id)
.map_or(SafetyContext::Safe, |fn_sig| {
    if fn_sig.header.safety == hir::Safety::Unsafe {
        SafetyContext::UnsafeFn
    } else {
        SafetyContext::Safe
    }
});
let body_target_features = &tcx.body_codegen_attrs(def.to_def_id()).target_features;
let mut warnings = Vec::new();
let mut visitor = UnsafetyVisitor {
    tcx, thir, .  .  ., warnings: &mut warnings,
    suggest_unsafe_block: true,
};
visitor.visit_expr(&thir[expr]);

warnings.sort_by_key(|w| w.block_span);
for UnusedUnsafeWarning { hir_id, block_span, enclosing_unsafe } in warnings {
    let block_span = tcx.sess.source_map().guess_head_span(block_span);
    tcx.emit_node_span_lint(
        UNUSED_UNSAFE,
        hir_id,
        block_span,
        UnusedUnsafe { span: block_span, enclosing: enclosing_unsafe },
    );}
\end{minted}
\caption{Провера безбедности}
\label{lst:safety_check}
\end{listing}


\section{Додатак 4}


\begin{listing}[H]
\begin{minted}{text}
fn main() -> () {
    let mut _0: ();
    let mut _1: std::vec::Vec<i32>;
    let _2: ();
    let mut _3: &mut std::vec::Vec<i32>;
    let _4: ();
    let mut _5: &mut std::vec::Vec<i32>;
    scope 1 {
        debug vec => _1;
    }
    bb0: {
        StorageLive(_1);
        _1 = Vec::<i32>::new() -> [return: bb1, unwind continue];
    }
    bb1: {
        StorageLive(_2);
        StorageLive(_3);
        _3 = &mut _1;
        _2 = Vec::<i32>::push(move _3, const 1_i32) -> [return: bb2, unwind: bb5];
    }
    bb2: {
        StorageDead(_3);
        StorageDead(_2);
        StorageLive(_4);
        StorageLive(_5);
        _5 = &mut _1;
        _4 = Vec::<i32>::push(move _5, const 2_i32) -> [return: bb3, unwind: bb5];
    }
    bb3: {
        StorageDead(_5);
        StorageDead(_4);
        _0 = const ();
        drop(_1) -> [return: bb4, unwind: bb6];
    }
    bb4: {
        StorageDead(_1);
        return;
    }
    bb5 (cleanup): {
        drop(_1) -> [return: bb6, unwind terminate(cleanup)];
    }
    bb6 (cleanup): {
        resume;
    }
}
\end{minted}
\caption{Неоптимизована МСН}
\label{lst:unoptimized-mir}
\end{listing}
