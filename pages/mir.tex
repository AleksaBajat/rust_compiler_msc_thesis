\subsection{Posredna Reprezentacija Srednjeg Nivoa - MIR (Mid-Level Intermediate Representation)}

Prvobitno koncept \verb|MIR|-a nije postojao. HIR se transformisao u LLVM reprezentaciju gde 
vršila optimizacija. Glavni povod za uvod reprezentacije srednjeg nivoa 
jeste inkrementalna kompilacija i MIR je struktuiran tako da ga je lako sačuvati i učitati, iako su se delovi
koda promenili tokom vremena.

U HIR-u optimizacija \verb|for| petlje je svedena na \verb|while(let)| u MIR-u se prevodi na \verb|loop match| 
primitivu \ref{lst:mir_for_1}.  Pozivi metoda su prevedeni u pozive funkcija još u THIR-u.

\begin{listing}[H]
\begin{minted}{rust}
let mut iterator = IntoIterator::into_iter(vec);
loop {
    match Iterator::next(&mut iterator){
        Some(elem) => process(elem),
        None => break,
    }
}
\end{minted}
\caption{"while let" posle pojednostavljenja}
\label{lst:mir_for_1}
\end{listing}

Reprezentacija srednjeg nivoa dozvoljava za još jedan vid optimizacije na nivou frontenda pojednostavljivanjem
sintakse na primitive koje nisu dostupne u regularnom jeziku.

\begin{listing}[H]
\begin{minted}{rust}
let mut iterator = IntoIterator::into_iter(vec);

loop:
    match Iterator::next(&mut iterator) {
        Some(elem) => { process(elem); goto loop; }
        None => { goto break; }
    }

break:
\end{minted}
\caption{"while let" posle pojednostavljenja}
\label{lst:mir_for_2}
\end{listing}

Izraz \verb|goto| je u inženjerskoj zajednici gledan sa visine jer kontrola toka nakon odredjene kompleksnosti
postaje nerazumljiva. Rust iz istog razloga ne dozvoljava upotrebu ove ključne reči, ali u MIR-u svodi 
petlju na najjednostavniji oblik \ref{lst:mir_for_2}.

\newpage
U isečku koda \ref{lst:mir_for_2}, jedina kompleksnija sintaktička celina je \verb|match|. U samom jeziku 
grupisanje provere i pristup podatku je smisleno ali u MIR-u se odvaja na dve celine \ref{lst:mir_for_3}.

\begin{listing}[H]
\begin{minted}{rust}
loop:
    let tmp = Iterator::next(&mut iterator);
    
    switch tmp {
        Some => {
            let elem = (tmp as Some).0;
            process(elem);
            goto loop;
        }
        None => {
            goto break;
        }
    }
    
break:
    ....
\end{minted}
\caption{"while let" posle pojednostavljenja}
\label{lst:mir_for_3}
\end{listing}

Naime ovako sveden kod nikada nije tekstualnom obliku. MIR je graf kontrole toka koji se može predstaviti
uz pomoć \verb|graphviz|-a \ref{lst:mir_print}. Filter mora biti zadatat prilikom poziva ove komande. 
Za prikaz celokupnog izlaza koristi se \verb|all| filter, dok ako je neka pojedinačna funkcija od interesovanja
njeno ime (mnogo češće slučaj).

\begin{listing}[H]
\begin{minted}{bash}
cargo rustc -- -Z dump-mir=[filter] -Z dump-mir-graphviz
\end{minted}
\caption{Ispis i prikaz MIR-a}
\label{lst:mir_print}
\end{listing}

\todo{MIR je opsežan, ne izgleda kao u primerima (umereni nivo obfuskacije). Ovde ima još značajno puno sadržaja koji nije opisan.}