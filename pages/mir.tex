\subsection{Medjureprezentacija srednjeg nivoa - MIR (Mid-Level Intermediate Representation)}

Koncept medjureprezentacije srednjeg nivoa nije postojao od samog nastanka kompajlera. Medjutim,
kako je kompajler postajao sve kompleksniji uvidela se potreba za reprezentacijom koja može da 
se upotrebi u kontekstu inkrementalne kompilacije, kao i potreba da ta medjureprezentacija lako manipuliše 
kontrolom toka. Kontrola toka je bila vezana za apstraktno sintaksno stablo što se pokazalo nedovoljno fleksibilno. 
Graf kontrole toka je pravi alat za ovaj posao jer je kontrola toka eksplicitna. Graf kontrole toka je 
struktuiran kao grupa osnovnih blokova koji su povezani granama. Ključna ideja bloka jeste da je to grupa naredbi (iskaza) 
koja se izvršava zajedno. Svaki put kada se granom dodje do novog bloka, sve naredbe tog bloka moraju da se izvrše, pa tek onda postoji mogućnost 
da se grana dalje. Poslednja nareba unutar bloka se naziva terminator. Brojni izrazi \ref{lst:block_code} koji se koriste u Rust jeziku se kompajliraju na 
više osnovnih blokova \ref{lst:block_block}.

\begin{listing}[H]
\begin{minted}{rust}
a = 1;
if some_variable {
    b = 1;
} else {
    c = 1;
}
d = 1;
\end{minted}
\caption{Isečak koda koji se prevodi u više osnovnih blokova}
\label{lst:block_code}
\end{listing}


\begin{listing}[H]
\begin{minted}{text}
BB0: {
    a = 1;
    if some_variable {
        goto BB1;
    } else {
        goto BB2;
    }
}
BB1: {
    b = 1;
    goto BB3;
}
BB2: {
    c = 1;
    goto BB3;
}
BB3: {
    d = 1;
    ...
}
\end{minted}
\caption{Isečak koda u formi osnovih blokova}
\label{lst:block_block}
\end{listing}



Ovo se pokazuje kao pravo mesto da se 
dealokacija memorije sprovede, svaki \verb|drop| je eksplicitan u srednjoj medjureprezentaciji. Jedina stvar koju 
srednja reprezentacija ne zna eksplicitno jeste da li se dogodio prenos vlasništva. Ovu informaciju dobija na osnovu 
tipa promenljive.

Dokazati bezbednost formalno je izuzetno težak 
problem u bilo kojem Turing kompletnom jeziku. Naime, reprezentacija srednjeg nivoa se generiše da bude dovoljno jednostavna 
da bi se u jednom trenutku formalni dokazi mogli pisati na račun nje. 

Optimizacije specifične za Rust su teško izvodljive bez uvodjenja reprezentacije pre interakcije sa LLVM-om. Srednja medjureprezentacija 
ima mogućnost da sprovede analizu da li je bezbedno primeniti optimizaciju, i ako jeste da je unapred primeni. Uvodjenjem srednje medjureprezentacije bolje povlači granica izmedju samog jezika i LLVM-a jer 
semantika jezika nije vezana za konverziju u LLVM medjureprezentaciju.


Medjureprezentacija srednjeg nivoa nastaje procesuiranjem izraza iz \verb|THIR| reprezentacije rekurzivno.
U HIR-u optimizacija \verb|for| petlje je svedena na \verb|while(let)| u MIR-u se prevodi na \verb|loop match| 
primitivu \ref{lst:mir_for_1}.  Pozivi metoda su prevedeni u pozive funkcija još u THIR-u.

\begin{listing}[H]
\begin{minted}{rust}
let mut iterator = IntoIterator::into_iter(vec);
loop {
    match Iterator::next(&mut iterator){
        Some(elem) => process(elem),
        None => break,
    }
}
\end{minted}
\caption{"while let" posle pojednostavljenja}
\label{lst:mir_for_1}
\end{listing}

Reprezentacija srednjeg nivoa dozvoljava za još jedan vid optimizacije na nivou frontenda pojednostavljivanjem
sintakse na primitive koje nisu dostupne u regularnom jeziku.

\begin{listing}[H]
\begin{minted}{rust}
let mut iterator = IntoIterator::into_iter(vec);

loop:
    match Iterator::next(&mut iterator) {
        Some(elem) => { process(elem); goto loop; }
        None => { goto break; }
    }

break:
\end{minted}
\caption{"while let" posle pojednostavljenja}
\label{lst:mir_for_2}
\end{listing}

Izraz \verb|goto| je u inženjerskoj zajednici gledan sa visine jer kontrola toka nakon odredjene kompleksnosti
postaje nerazumljiva. Rust iz istog razloga ne dozvoljava upotrebu ove ključne reči, ali u MIR-u svodi 
petlju na najjednostavniji oblik \ref{lst:mir_for_2}.

\newpage
U isečku koda \ref{lst:mir_for_2}, jedina kompleksnija sintaktička celina je \verb|match|. U samom jeziku 
grupisanje provere i pristup podatku je smisleno ali u MIR-u se odvaja na dve celine \ref{lst:mir_for_3}.

\begin{listing}[H]
\begin{minted}{rust}
loop:
    let tmp = Iterator::next(&mut iterator);
    
    switch tmp {
        Some => {
            let elem = (tmp as Some).0;
            process(elem);
            goto loop;
        }
        None => {
            goto break;
        }
    }
    
break:
    ....
\end{minted}
\caption{"while let" posle pojednostavljenja}
\label{lst:mir_for_3}
\end{listing}

Naime ovako sveden kod nikada nije tekstualnom obliku. MIR je graf kontrole toka koji se može predstaviti
uz pomoć \verb|graphviz|-a \ref{lst:mir_print}. Filter mora biti zadatat prilikom poziva ove komande. 
Za prikaz celokupnog izlaza koristi se \verb|all| filter, dok ako je neka pojedinačna funkcija od interesovanja
njeno ime (mnogo češće slučaj).

\begin{listing}[H]
\begin{minted}{bash}
cargo rustc -- -Z dump-mir=[filter] -Z dump-mir-graphviz
\end{minted}
\caption{Ispis i prikaz MIR-a}
\label{lst:mir_print}
\end{listing}

\todo{MIR je opsežan, ne izgleda kao u primerima (umereni nivo obfuskacije). Ovde ima još značajno puno sadržaja koji nije opisan.}