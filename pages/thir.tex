\subsection{Tipizirana medjureprezentacija visokog nivoa (TMVN) - Typed HIR (THIR)}

Tipizirana medjureprezentacija visokog nivoa je medjureprezentacija izvornog koda koja nastaje dopunom stabla tipovima.
Za razliku od \verb|HIR| medjureprezentacije, \verb|TMVN| sadrži samo tela tj. izvršni kod. To znači da \verb|TMVN| 
ne sadrži reprezentaciju stavki kao što su strukture (\verb|struct|) i osobine (\verb|traits|). Svako telo u ovoj reprezentaciji
se čuva priveremeno u memoriji i odbačeno čim više nije potrebno. Ovo je bitna distinkcija u donosu na \verb|HIR| gde se reprezentacija 
čuva tokom celokupnog procesa kompilacije. Dodatno, automatska referenciranja i dereferenciranja su eksplicitna, pozivi metoda i 
preklopljeni operatori su pretvoreni u obične pozive funkcija. Uništenje opsega je u ovoj reprezentaciji eksplicitno.
Izrazi, iskazi i klauzule \verb|match| odredbe se čuvaju posebno.

\subsubsection{Bezbednost}

Pojednostavljivanjem izvornog koda sažima se broj različitih načina da se isti kod napiše i smanjuje razdaljinu izmedju 
koda koji se analizira (AST) i koda koji se izvršava (bitcode).
Upravo taj manjak instrukcione kompleksnosti čini tipiziranu medjureprezentaciju značajnu u proveri bezbednosti. 
Provere bezbednosti se nalaze u modulu \verb|check_unsafety|.
Algoritam prolazi kroz telo funkcije i sve njene anonimne funkcije prateći da li je nebezbednom \verb|unsafe| kontekstu.
Ukoliko se nebezbedan kod poziva van nebezbednog (\verb|unsafe|) bloka greška će biti prikazana. Algoritam takodje vodi računa da li 
postoji nebezbedan blok u kome se ne koristi nebezbedan kod. Ako postoji ovakav blok kompajler će prikazati upozorenje (\verb|lint|).

Dodatak \ref{lst:safety_check} prikazuje isečak glavne funkcije koja je odgovorna za opseg bezbednosti. 
Na osnovu \verb|LocalDefId| dobavlja se telo u \verb|TMVN| medjureprezentaciji. Potom se na osnovu istog 
identifikatora evaluira \verb|HirId| uz pomoć kojeg se vrši upit nad \verb|HIR|-om čime se otkriva početni kontekst bezbednosti.
Struktura \verb|UnsafeVisitor| koristi \verb|TMVN| telo, na osnovu kog sprovodi prethodno objašnjen algoritam. Može se primetiti 
da se ovde incijalizuje vektor upozorenja koja se prikazuju ukoliko se nebezbedan kod ne koristi u nebezbednom kontekstu.

\subsubsection{Provera šablona}
Rust jezik poseduje karakteristiku da ukoliko se koristi provera šablona (\verb|pattern matching|), svaka moguća 
varijacija šablona mora biti obradjena (ima odgovarajuću logiku) tj. provera štablona je iscrpna \ref{lst:pattern_matching}.
Proveru da li je iscrpnost zadovoljena izvršava \verb|TMVN|. U provere šablona spadaju izrazi \verb|match|, \verb|if let|, \verb|while let|,
\verb|let else|, \verb|let|, pa čak i argumenti funkcija. Pored iscrpnosti, proverava se i korisnost šablona. Korisnost odgovara na pitanje 
da li je neko grananje redudantno. Ovo je vredno korisniku jezika jer skreće pažnju da je neki segment izvornog koda nedostižan. 



\begin{listing}[H]
\begin{minted}{rust}
pub enum IpAddrKind { V4, V6, }
fn main() {
    let x = IpAddrKind::V4;
    match x {
        IpAddrKind::V4 => println!("Ovo je IPv4 adresa."), 
        IpAddrKind::V6 => println!("Ovo je IPv6 adresa.") 
        // Da je postojao IpAddrKind::V7 Rust bi zahtevao 
        // da i ta varijanta bude obradjena.
    }
}
\end{minted}
\caption{Provera šablona}
\label{lst:pattern_matching}
\end{listing}