\subsection{Типизирана међурепрезентација високог \\ нивоа (ТМВН) - Typed HIR (THIR)}

Типизирана међурепрезентација високог нивоа је међурепрезентација изворног кода која настаје допуном стабла типовима.
За разлику од \verb|HIR| међурепрезентације, ТМВН садржи само тела тј. извршни код. То значи да ТМВН 
не садржи репрезентацију ставки као што су структуре (\verb|struct|) и особине (\verb|traits|). Свако тело у овој репрезентацији
се чува приверемено у меморији и одбачено чим више није потребно. Ово је битна дистинкција у доносу на \verb|HIR| где се репрезентација 
чува током целокупног процеса компилације. Додатно, аутоматска референцирања и дереференцирања су експлицитна, позиви метода и 
преклопљени оператори су претворени у обичне позиве функција. Уништење опсега је у овој репрезентацији експлицитно.
Изрази, искази и клаузуле \verb|match| одредбе се чувају посебно.

\subsubsection{Безбедност}

Поједностављивањем изворног кода сажима се број различитих начина да се исти код напише и смањује раздаљину између 
кода који се анализира (АСТ) и кода који се извршава (битцоде).
Управо тај мањак инструкционе комплексности чини типизирану међурепрезентацију значајну у провери безбедности. 
Провере безбедности се налазе у модулу \verb|check_unsafety|.
Алгоритам пролази кроз тело функције и све њене анонимне функције пратећи да ли је небезбедном \verb|unsafe| контексту.
Уколико се небезбедан код позива ван небезбедног (\verb|unsafe|) блока грешка ће бити приказана. Алгоритам такође води рачуна да ли 
постоји небезбедан блок у коме се не користи небезбедан код. Ако постоји овакав блок компајлер ће приказати упозорење (\verb|lint|).

Додатак \ref{lst:safety_check} приказује исечак главне функције која је одговорна за опсег безбедности. 
На основу \verb|LocalDefId| добавља се тело у ТМВН међурепрезентацији. Потом се на основу истог 
идентификатора евалуира \verb|HirId| уз помоћ којег се врши упит над \verb|HIR|-ом чиме се открива почетни контекст безбедности.
Структура \verb|UnsafeVisitor| користи ТМВН тело, на основу ког спроводи претходно објашњен алгоритам. Може се приметити 
да се овде инцијализује вектор упозорења која се приказују уколико се небезбедан код не користи у небезбедном контексту.

\subsubsection{Провера шаблона}
\verb|Rust| језик поседује карактеристику да уколико се користи провера шаблона (\verb|pattern matching|), свака могућа 
варијација шаблона мора бити обрађена (има одговарајућу логику) тј. провера штаблона је исцрпна \ref{lst:pattern_matching}.
Проверу да ли је исцрпност задовољена извршава ТМВН. У провере шаблона спадају изрази \verb|match|, \verb|if let|, \verb|while let|,
\verb|let else|, \verb|let|, па чак и аргументи функција. Поред исцрпности, проверава се и корисност шаблона. Корисност одговара на питање 
да ли је неко гранање редудантно. Ово је вредно кориснику језика јер скреће пажњу да је неки сегмент изворног кода недостижан. 

\begin{listing}[H]
\begin{minted}{rust}
pub enum IpAddrKind { V4, V6, }
fn main() {
    let x = IpAddrKind::V4;
    match x {
        IpAddrKind::V4 => println!("Ovo je IPv4 adresa."), 
        IpAddrKind::V6 => println!("Ovo je IPv6 adresa.") 
        // Da je postojao IpAddrKind::V7 Rust bi zahtevao 
        // da i ta varijanta bude obradjena.
    }}
\end{minted}
\caption{Провера шаблона}
\label{lst:pattern_matching}
\end{listing}